\documentclass[10pt,a4paper,twoside]{article}
\usepackage[utf8]{inputenc}
\usepackage[T1]{fontenc}
\usepackage{amsmath}
\usepackage[margin=2.54cm]{geometry}
\usepackage{amsfonts}
\usepackage{hyperref} % for link citations
\usepackage{amssymb}
\usepackage{makeidx}
\usepackage{graphicx}
\author{Valentin Nkana}
\title{PRELICA (Advanced methodologies for hydro-acoustics design in the naval propeller.)}
\begin{document}
	
\maketitle
%\tableofcontents
	
	
\section{Introduction.}
%The study of hydrodynamic noise is an active research field due to the rapid increase of noise levels in the oceans as well as its impact on maritime fauna. \cite{sezen2020numerical}

Generation and propagation of hydrodynamic noise in maritime environment is an active research field of investigation due to its own impact on several numbers of engineering applications including among others, the maritime propeller design (add.picture). 

A key point is the reduction of underwater noise induced by shipping activities to protect the maritime fauna (the ecosystem life). For instance, in \cite{sezen2020numerical}, preliminary results of a numerical study for noise prediction of a benchmark propeller in open water/ uniform flow conditions is presented. The main aim of this study is to predict propeller hydro-acoustic performance under cavity conditions.

\section{What's PRELICA?}

PRELICA (Advanced methodologies for hydro-acoustics design in the naval propeller) was a project co-financed by the European Regional Development Fund, Fruili Venezia Gulia (FVG) region located in Trieste (Italy) in 2014 -2020. The stressed on the development of innovative numerical tools and methodologies aimed at improving prediction of the underwater noise radiated by ships propeller since the early design stage. In this project two SMEs (Small and Medium-sized Enterprises) were involved: IFLUIDS and ENGYS. In addition, the University of Trieste, SISSA and the wide enterprise CETENA.

%The potsdam propeller(add a picture) test case and the data made

The starting point of this experimental research was the  potsdam propeller test case (PPTC) made available by the SVA (Schiffbau-Versuchsanstalt). Among many methodologies such as RANS, LES and DES. LES are recognized from scientific community as the best models for prediction of hydrodynamics noise especially for the one originated from  turbulence \cite{delafosse2008and}. Using OpenFOAM network, the research group IEFLUIDs of  the University of Trieste with his outstanding experience and expertise in LES methodology, developed the tools for the generation of simulations data for acoustic analysis and implementation of new acoustics models.  

In this project, three importance activities were carried out. Primary, the development of numerical Eddy-resolving techniques for the analysis of the hydrodynamic field generated by a ship propeller and more importantly, a cutting-edge mathematical model for the prediction of propeller noise along with the performance and maneuvering which is considered essential to help naval architects to design optimal ship holes and this requires accurate computational fluid dynamics (CFD) has been done by IFLUIDS. ENGYS () has developed new methods for the creation, computational grids, and advanced simulation solvers based on unsteady methods to correctly predict pressure fluctuations on the ships hall due to the propeller motion as well as underwater noise measured at arbitrary locations. SISSA () has done a lot of tests on many propeller to find for the one having the best hydro-acoustic performance. SISSA also developed instruments ot automatically deform an initial propeller shape (picture) and obtain as many variants as the tests required. A set of interesting mathematical tools which help evaluating the propeller performance in a fast way compared to usual simulation tools with no significant loss and predictions accuracy, have been developed. CETENA () a center owned by FICANTIERI Group since its foundation has always been involved in research and development of the new methodologies. In this project, CETENA evaluates their industrial application by comparing the results obtained by their partners with the outcomes of the methodologies currently in use in the industry. 

\section{Results and achievements.}

The main result obtained within PRELICA consists of an innovative design methodology obtained combining complementary high technology methods and tools.  It enables the engineers to find out and to describe the optimal layout of the propeller characterized by a reduced underwater hydro acoustic radiated noise. The project allows to validate this methodology, verify the accuracy of the models, and access their adaptability in the industry, the community itself will benefit from the result of the project that increases the possibility of designing ships, respecting the environmental requirements.



%\begin{thebibliography}{100}
%	\bibitem[label1]{cite_key1} bibliographic information
%	\bibitem[label2]{cite_key2} bibliographic information
%	...
%\end{thebibliography}	
\bibliographystyle{plain}
\bibliography{mybib}
\end{document}