\documentclass[12pt,a4paper,twoside]{article}
\usepackage[utf8]{inputenc}
\usepackage[T1]{fontenc}
\usepackage{amsmath}
\usepackage{amsfonts}
\usepackage{hyperref} % for link citations
\usepackage{amssymb}
\usepackage{makeidx}
\usepackage{graphicx}
\author{Valentin Nkana}
\title{PRELICA (Advanced methodologies for hydro-acoustics design in the naval propeller.)}
\begin{document}
	
\maketitle
%\tableofcontents
	
	
\section{Introduction.}
%The study of hydrodynamic noise is an active research field due to the rapid increase of noise levels in the oceans as well as its impact on maritime fauna. \cite{sezen2020numerical}

Generation and propagation of hydrodynamic noise in maritime environment is an active research field of investigation due to its own impact on several numbers of engineering applications including among others, the maritime propeller design (add.picture). A key point is the reduction of underwater noise induced by shipping activities to protect the maritime fauna (the ecosystem life). For instance, in \cite{sezen2020numerical}, preliminary results of a numerical study for noise prediction of a benchmark propeller in open water/ uniform flow conditions is presented. The main aim of this study is to predict propeller hydro-acoustic performance under cavity conditions.

\section{What's PRELICA?}

PRELICA (Advanced methodologies for hydro-acoustics design in the naval propeller) was a project co-financed by the European Regional Development Fund, Fruili Venezia Gulia (FVG) region located in Trieste (Italy) in 2014 -2020. The stressed on the development of innovative numerical tools and methodologies aimed at improving prediction of the underwater noise radiated by ships propeller since the early design stage.   



%\begin{thebibliography}{100}
%	\bibitem[label1]{cite_key1} bibliographic information
%	\bibitem[label2]{cite_key2} bibliographic information
%	...
%\end{thebibliography}	
\bibliographystyle{plain}
\bibliography{mybib}
\end{document}